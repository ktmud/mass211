\documentclass[]{tufte-handout}

% ams
\usepackage{amssymb,amsmath}

\usepackage{ifxetex,ifluatex}
\usepackage{fixltx2e} % provides \textsubscript
\ifnum 0\ifxetex 1\fi\ifluatex 1\fi=0 % if pdftex
  \usepackage[T1]{fontenc}
  \usepackage[utf8]{inputenc}
\else % if luatex or xelatex
  \makeatletter
  \@ifpackageloaded{fontspec}{}{\usepackage{fontspec}}
  \makeatother
  \defaultfontfeatures{Ligatures=TeX,Scale=MatchLowercase}
  \makeatletter
  \@ifpackageloaded{soul}{
     \renewcommand\allcapsspacing[1]{{\addfontfeature{LetterSpace=15}#1}}
     \renewcommand\smallcapsspacing[1]{{\addfontfeature{LetterSpace=10}#1}}
   }{}
  \makeatother

\fi

% graphix
\usepackage{graphicx}
\setkeys{Gin}{width=\linewidth,totalheight=\textheight,keepaspectratio}

% booktabs
\usepackage{booktabs}

% url
\usepackage{url}

% hyperref
\usepackage{hyperref}

% units.
\usepackage{units}


\setcounter{secnumdepth}{-1}

% citations
\usepackage{natbib}
\bibliographystyle{plainnat}

% pandoc syntax highlighting

% longtable
\usepackage{longtable,booktabs}

% multiplecol
\usepackage{multicol}

% strikeout
\usepackage[normalem]{ulem}

% morefloats
\usepackage{morefloats}


% tightlist macro required by pandoc >= 1.14
\providecommand{\tightlist}{%
  \setlength{\itemsep}{0pt}\setlength{\parskip}{0pt}}

% title / author / date
\title{Mass 211 Analysis: the Big Picture}
\author{Jesse Yang}
\date{2017-03-10}


\begin{document}

\maketitle




\section{Introduction}\label{introduction}

This report is a rough analysis on how 211 calls, including `Call2Talk'
calls, were made and handled, between \emph{Feb 25, 2016} and \emph{Feb
9, 2017}\footnote{If I understood the data correctly, the new system
  must be on a trial run before May, 2016, because that was when calls
  started steadily coming in.}. All the numbers and facts are for the
whole time span, i.e., they are about the whole landscape of human
service needs reached the Mass 211 system in this roughly 10 month
period.

\section{Key Findings}\label{key-findings}

\section{Drilling Down}\label{drilling-down}

\hypertarget{the-flow}{\subsection{The flow}\label{the-flow}}

Have you used Mass211 before?

How did you heard of Mass211?

AIRS I\&R Need Category

Referrals Made

Above graph shows where the calls were coming from and where the callers
were referred to. The majority of callers are first time callers
referred by human service agencies. \texttt{Housing} and
\texttt{Income\ Support/Assistance} are the two most inquired issues.
For \texttt{Income\ Support/Assistance} inquiries, most of them are
about early education and child care.

\subsection{Calls with multiple
purposes}\label{calls-with-multiple-purposes}

A call may have multiple purposes, and the caller can be referred to
multiple agencies. About one third of all 2-1-1 calls had more than one
Taxonomy term, and more than 47\% of the callers were given multiple
referrals.

\begin{longtable}[]{@{}lrr@{}}
\caption{Top 10 Level 3 Taxonomy term combinations}\tabularnewline
\toprule
Categories & Num. of Calls & \% of all calls\tabularnewline
\midrule
\endfirsthead
\toprule
Categories & Num. of Calls & \% of all calls\tabularnewline
\midrule
\endhead
Early Childhood Education ~ · Household Related Public Assistance
Programs & 435 & 1.54\tabularnewline
Consumer Protection Agencies ~ · Utility Assistance & 344 &
1.22\tabularnewline
Child Care Providers ~ · Household Related Public Assistance Programs &
242 & 0.86\tabularnewline
Family Based Services ~ · Household Related Public Assistance Programs &
231 & 0.82\tabularnewline
Early Childhood Education ~ · Family Based Services ~ · Household
Related Public Assistance Programs & 213 & 0.76\tabularnewline
Emergency Shelter ~ · Housing Search and Information & 195 &
0.69\tabularnewline
Consumer Complaints ~ · Utility Assistance & 167 & 0.59\tabularnewline
Housing Expense Assistance ~ · Housing Search and Information & 163 &
0.58\tabularnewline
Housing Expense Assistance ~ · Utility Assistance & 148 &
0.52\tabularnewline
Basic Income Maintenance Programs ~ · Household Related Public
Assistance Programs & 136 & 0.48\tabularnewline
\bottomrule
\end{longtable}

The most common case of multiple purpose calls were calls asking for
child care assistance programs, then information regarding child care
centers or early start sites were also provided.

Another case is attempting to make complaints about utility companies
while seeking assistance in utility payment.

\subsection{Calls with single purpose}\label{calls-with-single-purpose}

\begin{longtable}[]{@{}lrr@{}}
\caption{Top 10 Level 5 Taxonomy terms for single purpose
call}\tabularnewline
\toprule
Taxonomy Term & Num. of Calls & \% of all calls\tabularnewline
\midrule
\endfirsthead
\toprule
Taxonomy Term & Num. of Calls & \% of all calls\tabularnewline
\midrule
\endhead
Child Care Expense Assistance & 8116 & 28.78\tabularnewline
Child Care Expense Assistance Applications & 1233 & 4.37\tabularnewline
Homeless Shelter & 1029 & 3.65\tabularnewline
211 Systems & 1002 & 3.55\tabularnewline
Rent Payment Assistance & 989 & 3.51\tabularnewline
Electric Service Payment Assistance & 821 & 2.91\tabularnewline
Directory Assistance & 407 & 1.44\tabularnewline
Food Pantries & 275 & 0.98\tabularnewline
Food Stamps/SNAP Applications & 259 & 0.92\tabularnewline
Holiday Gifts/Toys & 194 & 0.69\tabularnewline
\bottomrule
\end{longtable}

Child care related calls took the largest portion of the caseload, but
it is only because some parents frequently called to check status of
their enrollment application for EEC subsities. \footnote{Ref:
  \href{http://www.mass.gov/edu/birth-grade-12/early-education-and-care/}{Early
  Education and Care}. About 20\% of child care related calls were
  explicitly marked as status check.}

``211 Systems'' are mostly calls from neighboring states, e.g.,
Conneticut and Rhode Island; ``Directory Assistance'' are those intended
to call 411. Putting these categories aside, the top demands are EEC
Applications, Homeless Shelter or Rent Payment, Electricity, and Food.

\subsection{Call length}\label{call-length}

The average call length is 7.8 minutes, and most calls (75\%) would
finish within 10 minutes, 95\% finish within 20 minutes. Some took as
long as 40 minutes or more, but those were rare cases and often involved
interpretation services (the caller didn't speak English).

When removed undesired calls (211 Systems and Directory Assistance)
which normally end fast, the avergae call length becomes 8.3 minutes.

\begin{longtable}[]{@{}rrrrrr@{}}
\caption{Mean, median and quantiles of call length}\tabularnewline
\toprule
minimum & q1 & median & mean & q3 & maximum\tabularnewline
\midrule
\endfirsthead
\toprule
minimum & q1 & median & mean & q3 & maximum\tabularnewline
\midrule
\endhead
0 & 3 & 6 & 7.834 & 10 & 403\tabularnewline
\bottomrule
\end{longtable}

Calls checking EEC status have a slightly higher average call length of
10.2 minutes.

If taking call length into consideration, the top case load types are:

\section{Notes}\label{notes}

This report is based on the iCarol reports generated on Feb 9, 2017. The
data were manually cleaned and transformed, with a few noteworthy
controls:

\begin{enumerate}
\def\labelenumi{\arabic{enumi}.}
\tightlist
\item
  Only regular Mass 211 calls are used\footnote{ReportVersion = Mass 211}.
  ``Call2Talk'', ``Runaway Form'', and ``Mass 211 Text'', ``Mass 211
  Chat'' were all ignored, the reason being that their number are small
  and they often contain incomplete or unique data fields, which means
  some separate analyses are more appropriate.
\item
  Different data fields about the same information were merged into one.
  For example, ``Data Collection - How did you Learn about Mass211'' and
  ``Caller Data - How Heard about Call2Talk?'' are treated as one field.
\item
  Some missing records of AIRS Problem/Needs categories are filled by
  implications of the Taxonomy. E.g., ``Target People -\textgreater{}
  Low Income'' was mapped to ``Income Support/Assistance''.
\end{enumerate}

\subsection{Categorization}\label{categorization}

\href{http://www.airs.org/i4a/pages/index.cfm?pageid=3386}{The Taxonomy}
provides a comprehensive and logical structure for human services, but
is not suitable for public communication and strategic
planning--end-level terms too granular, and the upper levels too broad.

The
\href{https://www.acallforhelp.info/cms6/files/os007/p46/AIRS\%20ProblemNeeds\%20definitions\%202012.pdf}{AIRS:
I\&R Problem/Needs National Categories}\footnote{The linked document is
  outdated. It contains only 16 categories, but AIRS has
  \href{http://www.icarol.com/changes-to-airs-problemsneeds-categories/}{split}
  ``Housing/Utility'' into two separate categories in 2014, making it 17
  categories in our case.} recorded in the \texttt{MetUnmet} report is a
better candidate for reporting, and was used in
\protect\hyperlink{the-flow}{the flow chart} at the beginning fo this
document, but they seem not revealing enough. A flat, topic-based
categorization method is desirable if we want a more readable report.

\section{Questions and Additional Data
Request}\label{questions-and-additional-data-request}

\begin{enumerate}
\def\labelenumi{\arabic{enumi}.}
\tightlist
\item
  Are resource agencies categorized in any way in the system? E.g.,
  tagged as Police, Municipality Service, Government Aide Progarm,
  Volunteer and Charity, Hostpital, Shelters, etc. It would be
  interesting to see request fullfilment and resource dispatch in a more
  generalized fashion.
\end{enumerate}

\bibliography{skeleton.bib}



\end{document}
